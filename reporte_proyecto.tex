\documentclass[12pt, a4paper]{article}
\usepackage[utf8]{inputenc}
\usepackage[spanish]{babel}
\usepackage{graphicx}
\usepackage{hyperref}
\usepackage{listings}
\usepackage{xcolor}
\usepackage{geometry}

\geometry{top=2.5cm, bottom=2.5cm, left=2.5cm, right=2.5cm}

\definecolor{codegreen}{rgb}{0,0.6,0}
\definecolor{codegray}{rgb}{0.5,0.5,0.5}
\definecolor{codepurple}{rgb}{0.58,0,0.82}
\definecolor{backcolour}{rgb}{0.95,0.95,0.92}

\lstdefinestyle{mystyle}{
    backgroundcolor=\color{backcolour},   
    commentstyle=\color{codegreen},
    keywordstyle=\color{magenta},
    numberstyle=\tiny\color{codegray},
    stringstyle=\color{codepurple},
    basicstyle=\ttfamily\footnotesize,
    breakatwhitespace=false,         
    breaklines=true,                 
    captionpos=b,                    
    keepspaces=true,                 
    numbers=left,                    
    numbersep=5pt,                  
    showspaces=false,                
    showstringspaces=false,
    showtabs=false,                  
    tabsize=2
}

\lstset{style=mystyle}

\title{Reporte de Implementación: Sistema de Soporte a Decisiones (DSS)}
\author{Equipo de Desarrollo}
\date{\today}

\begin{document}

\maketitle

\begin{abstract}
Este documento detalla la implementación de un Sistema de Soporte a Decisiones (DSS) diseñado para optimizar la gestión de proyectos de software. El sistema integra un Dashboard estratégico (Balanced Scorecard), un modelo predictivo de calidad basado en simulaciones de Montecarlo y distribución de Rayleigh, y un proceso ETL automatizado, cumpliendo con los estándares de calidad y seguridad requeridos.
\end{abstract}

\tableofcontents
\newpage

\section{Introducción}
El objetivo del proyecto es desarrollar una herramienta que permita a la alta gerencia y a los Project Managers tomar decisiones informadas basadas en datos históricos y proyecciones futuras. El sistema se alinea con la misión de la empresa de fomentar la innovación, la trazabilidad y el uso ético de los datos.

\section{Arquitectura del Sistema}
El sistema está construido utilizando un stack tecnológico moderno y eficiente:
\begin{itemize}
    \item \textbf{Lenguaje}: Python 3.9+
    \item \textbf{Frontend/Backend}: Streamlit (Framework para Data Apps)
    \item \textbf{Procesamiento de Datos}: Pandas y NumPy
    \item \textbf{Visualización}: Plotly Express y Graph Objects
    \item \textbf{Almacenamiento}: Archivos CSV estructurados (ROLAP ligero)
\end{itemize}

\section{Procesos ETL (Extracción, Transformación y Carga)}
El flujo de datos garantiza que la información mostrada en el dashboard sea precisa y actualizada.
\begin{enumerate}
    \item \textbf{Extracción}: Se leen datos crudos de transacciones y registros de proyectos.
    \item \textbf{Transformación}: Se limpian los datos, se calculan KPIs financieros (márgenes, costos) y de calidad (densidad de defectos).
    \item \textbf{Carga}: Los datos procesados se almacenan en \texttt{OLAP\_Proyectos.csv} y \texttt{OLAP\_Calidad.csv}.
\end{enumerate}

\section{Componentes del DSS}

\subsection{Dashboard Estratégico (Balanced Scorecard)}
El tablero principal ofrece una visión holística del desempeño de la empresa, dividida en perspectivas:
\begin{itemize}
    \item \textbf{Financiera}: Visualización de Ganancia Neta, Margen de Beneficio y Costos Operativos.
    \item \textbf{Clientes}: Análisis de rentabilidad por cliente y distribución geográfica (Mapa de Calor).
    \item \textbf{Procesos Internos}: Monitoreo de defectos de calidad y su severidad.
\end{itemize}

\subsection{Modelo Predictivo de Calidad}
Para anticipar riesgos en futuros proyectos, se implementó un modelo estocástico avanzado.

\subsubsection{Simulación de Montecarlo}
Utilizando los datos históricos de defectos por proyecto, el sistema calcula la media ($\mu$) y desviación estándar ($\sigma$). Con estos parámetros, se ejecutan miles de simulaciones (configurable entre 1,000 y 10,000) para estimar el rango probable de defectos totales en el próximo proyecto.

\subsubsection{Curva de Rayleigh}
La distribución de los defectos en el tiempo se modela utilizando la curva de Rayleigh, estándar en ingeniería de software para la confiabilidad:
\begin{equation}
    f(t) = \frac{2K}{t_m} \frac{t}{t_m} e^{-(t/t_m)^2}
\end{equation}
Donde $K$ es el volumen total de defectos simulado y $t_m$ es el tiempo donde ocurre el pico de defectos.

\begin{lstlisting}[language=Python, caption=Implementación de Rayleigh en Python]
# Formula Rayleigh PDF para Software
defectos_t = (2 * K / tm) * (t / tm) * np.exp(-(t/tm)**2)
\end{lstlisting}

\section{Seguridad y Control de Acceso}
El sistema implementa un control de acceso basado en roles (RBAC).
\begin{itemize}
    \item \textbf{Autenticación}: Manejo seguro de credenciales mediante \texttt{secrets.toml} y gestión de estado de sesión.
    \item \textbf{Autorización}: El módulo de predicción (Tab 2) está restringido exclusivamente a usuarios con rol de 'admin' o 'pm'. Los usuarios invitados no tienen acceso a esta funcionalidad sensible.
\end{itemize}

\section{Conclusión}
El sistema DSS implementado cumple con todos los requerimientos funcionales y no funcionales. Provee una interfaz intuitiva para el monitoreo del negocio y una herramienta potente para la gestión proactiva de la calidad, soportada por una arquitectura de datos sólida y documentada.

\end{document}
